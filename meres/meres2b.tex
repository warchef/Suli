\documentclass[10pt,a4paper]{article}
\usepackage[utf8]{inputenc}
\usepackage[magyar]{babel}
\usepackage[T1]{fontenc}
\usepackage{amsmath}
\usepackage{amsfonts}
\usepackage{amssymb}
\usepackage{graphicx}
\begin{document}
\title{Méréstechnika laboratórium 2/b jegyzőkönyv}
\author{Koncz István Márton}
\date{\today}
\maketitle
\newpage
\section{16. sz. laboratóriumi mérés}
	\subsection{A mérés célja}
	Egyes nem-villamos fizikai jellemzők (erő, nyomaték, nyomás, mechanikai feszültség)
mérésére alkalmas nyúlásmérő bélyeg fontosabb statikus méréstechnikai jellemzőinek
megállapítása. Egy adott feladatra való alkalmazás megismerése. A mérést “zavaró” jellemzők közül a hőmérsékletváltozás hatásának, mértékének megállapítása, vizsgálata.
	\subsection{Mérési feladatok}
	A mérésben egy befogott rugólap átellenes oldalára ragasztott mérőbélyegekkel állapíthatjuk
meg a lapban keletkező mechanikai feszültség értékét. A rugólap hajlítását csavarorsóval
hozzuk létre.
A hajlítás mértékét - az elmozdulást ($\Delta f$) 1/100 mm-es mérőórával mérjük.
A csavarorsó tengelyében ható F erő L (43 mm) hosszúságú karon végzi a hajlítást, a rugólap
keresztmetszeti méretei: h (1,2 mm) * b (10 mm.)
		\subsubsection{Elmozdulás (hajlítás) - ellenállásváltozás karakterisztika felvétele és kiértékelése}
		A 6. sz. mérőpanelon a felső kivezetések a húzott, az alsók a nyomott bélyeghez csatlakoznak.
Mérje meg mindkét bélyeg ellenállását a rugólap terheletlen és maximális kitérése között 0,5
mm-enként. A mért adatokat foglalja táblázatba és ábrázolja eltolt koordináta rendszerben mm papíron! (Az eltolás mértéke a bélyegek előfeszítés nélküli alap ellenállása.)
Állapítsa meg a húzott és a nyomott bélyegek linearitási és hiszterézis hibáját!$$$$A linearitási hiba megállapítása:$$h_{lin}=\frac{H_{max}}{X_kMT}*100 \%$$Mért adatok húzott bélyeg esetén:$$$$\begin{tabular}{|c|c|c|c|c|c|c|c|}
\hline 
l[mm] & 0 & 0,5 & 1 & 1,5 & 2 & 2,5 & 3 \\ 
\hline 
R[$\Omega$] & 354,41 & 354,52 & 354,62 & 354,69 & 354,70 & 354,88 & 354,97 \\ 
\hline 
\end{tabular} $$$$
Mért adatok húzott bélyeg esetén (visszafelé):$$$$\begin{tabular}{|c|c|c|c|c|c|c|c|}
\hline 
l[mm] & 3 & 2,5 & 2 & 1,5 & 1 & 0,5 & 0 \\ 
\hline 
R[$\Omega$] & 354,97 & 354,88 & 354,79 & 354,72 & 354,63 & 354,56 & 354,48 \\ 
\hline 
\end{tabular}$$$$
Mért adatok nyomott bélyeg esetén:$$$$\begin{tabular}{|c|c|c|c|c|c|c|c|}
\hline 
l[mm] & 0 & 0,5 & 1 & 1,5 & 2 & 2,5 & 3 \\ 
\hline 
R[$\Omega$] & 343,64 & 343,53 & 343,43 & 343,34 & 343,24 & 343,14 & 343,04 \\ 
\hline 
\end{tabular} $$$$ A számolás és az ábra az 1. számú mellékletben található meg.
	\subsubsection{Nyúlásmérő bélyeges negyedhíd vizsgálata!}
	Mérje meg a híd kimeneti feszültségét a rugólap terheletlen és maximális kitérése között 0,5 mm-enként. A hídban történő mérést is mindkét bélyeggel (húzott és nyomott) külön-külön végezze el! (Nekünk csak a húzott bélyeggel kellett.)
Az egyik bélyegnél az elmozdulás csökkentésekor (“visszafelé”) is vegye fel az adatokat!
A hídban lévő R ellenállások értéke: 360 $\pm 1\%$.
Foglalja a mért adatokat táblázatba és rajzolja meg a hídkapcsolás Uki-
(elhajlítás) karakterisztikáját! (Mivel a híd nincsen kinullázva az
ábrázolást eltolt koordináta rendszerben végezze. Az eltolás mértéke a híd
alapállapotban mért kimeneti feszültsége legyen.)
Állapítsa meg a kapcsolás átalakítási tényezőjét!$$$$
Mért adatok húzott bélyeg esetén:$$U_0 = 25,16 mV$$
\begin{tabular}{|c|c|c|c|c|c|c|c|}
\hline 
l[mm] & 0 & 0,5 & 1 & 1,5 & 2 & 2,5 & 3 \\ 
\hline 
$\Delta U$[$mV$] & 0 & -0,35 & -0,69 & -1,01 & -1,33 & -1,63 & -1,96 \\ 
\hline 
\end{tabular} $$$$
Mért adatok húzott bélyeg esetén (visszafelé):$$$$
\begin{tabular}{|c|c|c|c|c|c|c|c|}
\hline 
l[mm] & 3 & 2,5 & 2 & 1,5 & 1 & 0,5 & 0 \\ 
\hline 
$\Delta U$[$mV$] & -1,96 & -1,60 & -1,27 & -0,95 & -0,63 & -0,32 & 0 \\ 
\hline 
\end{tabular} $$$$
A számolás és az ábra az 2. számú mellékletben található meg.
	\subsubsection{Nyúlásmérő bélyeges félhíd vizsgálata!}
	Mérje meg a híd kimeneti feszültségét a rugólap terheletlen és maximális kitérése között 0,5 mm-enként. Foglalja a mért adatokat táblázatba és rajzolja meg a hídkapcsolás Uki- (elhajlítás) karakterisztikáját! (A karakterisztikát az előző ábrába rajzolja bele!)
Értékelje a kapcsolás érzékenységét a negyedhídhoz képest!
$$$$
Mért adatok húzott bélyeg esetén:$$U_0 = 42 mV$$
\begin{tabular}{|c|c|c|c|c|c|c|c|}
\hline 
l[mm] & 0 & 0,5 & 1 & 1,5 & 2 & 2,5 & 3 \\ 
\hline 
$\Delta U$[$mV$] & 0 & 0,8 & 1,6 & 2,2 & 3,0 & 3,7 & 4,5 \\ 
\hline 
\end{tabular} $$$$ Az ábra az 2. számú mellékletben található meg.
	\subsubsection{A hőmérsékleti hatás vizsgálata!}
	A rugólap terhelését állítsa be a maximális 3 mm-re.
Mérje meg a húzott és a nyomott mérőbélyeg ellenállását.
A mérőpanelba épített hőfokszabályzós fűtőtest segítségével a mérőbélyegek tere (és így a
mérőbélyegek hőmérséklete) kb. 44 $^{\circ}C$-ra beállítható, így a hőmérsékleti hatások
vizsgálhatók.
Kapcsoljon 12 V feszültséget a FŰTÉS feliratú pontra!
Mintegy 15 perc múlva beáll a termikus egyensúly. Ilyenkor a tér hőmérséklete kb. 44$^{\circ}C$.
A hőmérsékletváltozás hatására megváltozik a mérőbélyeg ellenállása.
A tér hőmérsékletének elérését a fűtő áram értékének lecsökkenése jelzi.
A tér felfűtött állapotában ismét mérje meg a húzott és a nyomott mérőbélyeg ellenállását!
Az ellenállás mérést kétféle „polaritással” végezze el! Értékelje a kapott értékeket!
$$$$Húzott bélyeg:$$$$\begin{tabular}{|c|c|c|}
\hline 
• & Fűtés nélkül & Fűtve \\ 
\hline 
Polaritás helyesen & 354,44 $\Omega$ & 353,82 $\Omega$ \\ 
\hline 
Ellentétesen & 354,44 $\Omega$ & 355,97 $\Omega$ \\ 
\hline 
\end{tabular} 
$$$$Fűtve 1$\%$ különbség van, ha ellentétes polaritással kötjük be a bélyeget.

	\section{17. sz. laboratóriumi mérés}
	\subsection{A mérés célja}
	A digitális oszcilloszkóp kezelésének többlet funkcióinak
elsajátítása, a kapott mérési eredmények kiértékeléséhez
szükséges szemlélet kialakítása.
	\subsection{Mérési feladatok}
	\subsubsection{Kurzoros mérések}
	Végezzen kurzoros méréseket szinuszos jel frekvenciájának
és amplitúdójának, illetve négyszögjel felfutási idejének
megállapításához, hasonlítsa össze mérési eredményeit az
azonos jelalakon elvégzett automatikus mérések
eredményeivel. (Javasolt jelalakok: 1kHz-es, 1V amplitúdójú
szinuszjel, 1V DC ofszettel, 10kHz-es 1V amplitúdójú
négyszögjel, 100mV DC ofszettel.) Mindenképpen térjen ki a
jelek effektív értékének automatikus mérésére, a kapott
eredményeket hasonlítsa össze a kurzoros mérések alapján
számított értékekkel!$$\begin{tabular}{|c|c|c|}
\hline 
Jelalak & Szinusz & Négyszög \\ 
\hline 
 Amplitúdó (kurzor) & 2 V & 2 V \\ 
\hline 
Amplitúdó (automatikus) & 2,03 V & 2 V \\ 
\hline 
 Felfutási idő (kurzor) & 1 ms & 79,6 ns \\ 
\hline 
Felfutási idő (automatikus) & 1 ms & 81 ns \\ 
\hline 
\end{tabular} $$ 
 \subsubsection{Adatgyűjtési mód}
 	Vizsgálja a függvénygenerátoron a legnagyobb beállítható
frekvenciájú négyszögjel egy periódusát, majd csak a felfutó
élét a három adatgyűjtési üzemmódban! Rajzolja le a látott
jelalakokat! $$$$ Az ábra a 3. számú mellékletben található meg.
	\subsubsection{Bemeneti komparátor működési feltételeinek, működési idejének vizsgálata}
	A LEVEL potencióméterrel a komparálási szintet állítja a SENSE
potencióméterrel pedig a hiszterézis nagyságát.
Adjon az 3. mérőpanelre +5V tápfeszültséget. Állítson be a
függvénygenerátoron 2V csúcsértékű 1kHz frekvenciájú háromszögjelet 0V
egyenfeszültségű összetevővel, a jelet csatlakoztassa a mérőpanel CH1
bemenetére! Kétcsatornás oszcilloszkóp egyik bemenetére a CH1 jelét, másik bemenetére
a KOMP1 komparátor kimeneti jelét csatlakoztatva mérje meg a
következőket: $$$$
	Milyen határok közt tudja állítani a komparálási szintet a LEVEL
potenciométerrel? A SENSE potencióméter a jobb oldali végállásában
legyen! $$0<x<1,12 V$$
	Mekkora a maximálisan beállítható hiszterézis a SENSE potencióméterrel?
	$$min -> \Delta t 200 \mu s$$
	$$max -> \Delta t 240 \mu s$$
	Az oszcilloszkóp XY üzemmódjának segítségével rajzolja le a
hiszterézises komparátor $U_{ki}-U_{be}$ karakterisztikáját! A SENSE
potencióméter a bal míg a LEVEL potencióméter a jobboldali
végállásban legyen.$$$$ Az ábra a 4. számú mellékletben található meg.
\subsubsection{Komparátor késleltetési idejének vizsgálata}
	Az előbbi kapcsolást felhasználva adjon a CH1 bemenetre 100 kHz-es
négyszögjelet. A négyszög-jel negatív csúcsértéke 0 V, pozitív csúcsértéke 3
V legyen.
Mérje meg a komparátor késleltetési idejét a négyszögjel fel- és lefutó éléhez
képest! $$fel -> \Delta t 400 ns$$
		$$le -> \Delta t 920 ns$$
	\section{18. sz. laboratóriumi mérés}
	\subsection{A mérés célja}
	Az ipari méréstechnikában a leggyakoribb mérendő jellemző a
hőmérséklet.
Hőmérséklet mérésére széles hőmérséklet tartományban
fémalapú mérőellenállásokat, kisebb hőmérsékleti tartomány, de
nagy érzékenységi igény esetén a félvezető alapúakat
(termisztorok) alkalmaznak. Egyre elterjedtebbek az analóg vagy
digitális kimeneti jellel rendelkező hőmérséklet mérő chippek is.
Nagyobb hőmérsékletek mérésekor (0 – 1600 $^{\circ}C$) hőelemeket
használnak. Jelen mérésben az említett hőmérséklet érzékelők legfontosabb
tulajdonságaival ismerkedünk meg.
	\subsection{Mérési feladatok}
\subsubsection{Önmelegedés vizsgálata}
Mérje meg a Therm 0 termisztor ellenállását a HM8012-es típusú
digitális multiméterrel az alábbi táblázatban megadott
méréshatárokban.
Mint az alábbi táblázatból is látja a DMM az ellenállásmérő
üzemmód esetén a különböző méréshatárokban más és más
áramot hajt át a mérendő ellenálláson.
A különböző méréshatárok beállítása után várja meg a termikus
egyensúly beállását (nem változik tovább a mért ellenállás kb. 2
perc). A 10 $\mu A$-es mérőáram esetén a mérés kezdetén mért értéket
tekintse alap értéknek. Számítsa ki, hogy a különböző mérőáramok a termikus egyensúly
beállta után mekkora relatív hibát okoznak!$$$$Mért értékek:$$$$\begin{tabular}{|c|c|c|c|c|c|}
\hline 
Méréshatár & Mérőáram & R (mérés kezdetén) & R (termikus egyensúly beállta után) & $\Delta R$ & $h \%$ \\ 
\hline 
500 $k\Omega$ (L4) & 10 $\mu A$ & 11,44 $k\Omega$ & 11,31 $k\Omega$ & 120 $\Omega$ & • \\ 
\hline 
50 $k\Omega$ (L3) & 100 $\mu A$ & 11,44 $k\Omega$ & 11,18 $k\Omega$ & 260 $\Omega$ & • \\ 
\hline 
5 $k\Omega$ (L2) & 1 mA & 11,44 $k\Omega$ & Offline & - & - \\ 
\hline 
\end{tabular} 
\subsubsection{Hőmérséklet jelleggörbék felvétele}
A mérendő objektum egy 6x6x1cm-es alumínium tömb, melynek
a hátlapjára (6x6cm) egy 47 $\Omega$-os 25 W-os fűtőellenállást
helyeztünk el.
Az érzékelők a tömb felső szélén vannak elhelyezve. Ezzel az
elrendezéssel próbáljuk meg biztosítani, hogy minden érzékelő
közel azonos hőmérsékletet mérjen.
Az alumínium tömb tetején a 6x1 cm-es felületen van 4 féle
érzékelő rögzítve, ezek kivezetéseit találják meg a mérőpanel
előlapján.
A mérőpanel tápfeszültség ellátásának egy része fixen bekötött
(12V és 42V), de az előlapra egy $\pm$15 V-os tápfeszültséget kell
csatlakoztatni.
A hőmérséklet változtatását a beépített kapcsoló változtatásával
tudják elérni.
A maximális hőmérséklet biztonsági okokból kb 50 $^{\circ}C$
A mérést a hőmérséklet beállító kapcsoló minden állásában
végezze el. A hőmérséklet beállítását egy szabályzás végzi. A
kapcsolóval az alapjelet állítják. A stabil hőmérséklet
beállításához kb. 3 percre van szükség. A termikus egyensúly
beállásának elérését a thermisztor ellenállásváltozását figyelve
tudjuk legegyszerűbben nyomon követni. Amikor a thermisztor
ellenállása már nem változik (esetleg csak nagyon lassan)
beálltnak tekinthetjük a termikus egyensúlyt.
A méréshez használjon két darab HM8012-es, egy TR1667/B és
két MX-25201-es típusú digitális multimétert.
Tervezze meg melyik műszerrel érdemes melyik érzékelő kimeneti
jelét mérni! A panel bal oldalán lévő piros banánhüvelyekre csatlakoztassa az
egyik HM8012-es multimétert $^{\circ}C$ mérő üzemmódban, a műszer
által mutatott feszültséget tekintsük a „hiteles” referenica
hőmérsékletnek!
$$$$Mért értékek:$$$$\begin{tabular}{|c|c|c|c|c|}
\hline 
Referencia hőmérséklet [$^{\circ}C$] & Therm [k$\Omega$] & PT100 [$\Omega$] & Hőelem [mV] & IC [mV] \\ 
\hline 
28 & 7,68 & 108,6 & 0,1 & 144 \\ 
\hline 
29,1 & 7,32 & 109,2 & 0,2 & 271,19 \\ 
\hline 
34,1 & 5,81 & 111,7 & 0,3 & 294,76 \\ 
\hline 
36,9 & 5,11 & 113,8 & 0,3 & 311,82 \\ 
\hline 
40,1 & 4,43 & 114,9 & 0,3 & 328,73 \\ 
\hline 
43 & 3,9 & 115,9 & 0,3 & 346,7 \\ 
\hline 
46,4 & 3,35 & 117,2 & 0,3 & 364,11 \\ 
\hline 
\end{tabular} $$$$Az ábra az 5. számú mellékletben található meg.
\subsubsection{A szilárd testek felületén mérhető hőmérséklet eloszlásának vizsgálata}
A mérendő objektum felfűtés után a kézi tapintófejes
hőmérsékletmérő műszer segítségével mérje meg az alumínium
tömb közepén és alsó szélén a hőmérsékletét.
Adjon magyarázatot a mért hőmérséklet értékekre.$$$$Mért értékek:$$$$\begin{tabular}{|c|c|}
\hline 
Középen & Alul \\ 
\hline 
38,4 $^{\circ}C$ & 39,5 $^{\circ}C$  \\ 
\hline 
\end{tabular} 
\tableofcontents
\end{document}