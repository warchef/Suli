\documentclass[10pt,a4paper]{article}
\usepackage{anysize}
\paperwidth=21cm \paperheight29.7cm
\marginsize{1cm}{1cm}{1cm}{1cm}
\usepackage[utf8]{inputenc}
\usepackage[magyar]{babel}
\usepackage[T1]{fontenc}
\usepackage{amsmath}
\usepackage{amsfonts}
\usepackage{amssymb}
\usepackage{graphicx}
\begin{document}
\section{18. sz. laboratóriumi mérés}
	\subsection{A mérés célja}
	Az ipari méréstechnikában a leggyakoribb mérendő jellemző a
hőmérséklet.
Hőmérséklet mérésére széles hőmérséklet tartományban
fémalapú mérőellenállásokat, kisebb hőmérsékleti tartomány, de
nagy érzékenységi igény esetén a félvezető alapúakat
(termisztorok) alkalmaznak. Egyre elterjedtebbek az analóg vagy
digitális kimeneti jellel rendelkező hőmérséklet mérő chippek is.
Nagyobb hőmérsékletek mérésekor (0 – 1600 $^{\circ}C$) hőelemeket
használnak. Jelen mérésben az említett hőmérséklet érzékelők legfontosabb
tulajdonságaival ismerkedünk meg.
	\subsection{Mérési feladatok}
\subsubsection{Önmelegedés vizsgálata}
Mérje meg a Therm 0 termisztor ellenállását a HM8012-es típusú
digitális multiméterrel az alábbi táblázatban megadott
méréshatárokban.
Mint az alábbi táblázatból is látja a DMM az ellenállásmérő
üzemmód esetén a különböző méréshatárokban más és más
áramot hajt át a mérendő ellenálláson.
A különböző méréshatárok beállítása után várja meg a termikus
egyensúly beállását (nem változik tovább a mért ellenállás kb. 2
perc). A 10 $\mu A$-es mérőáram esetén a mérés kezdetén mért értéket
tekintse alap értéknek. Számítsa ki, hogy a különböző mérőáramok a termikus egyensúly
beállta után mekkora relatív hibát okoznak!$$$$Mért értékek:$$$$\begin{tabular}{|c|c|c|c|c|c|}
\hline 
Méréshatár & Mérőáram & R (mérés kezdetén) & R (termikus egyensúly beállta után) & $\Delta R$ & $h \%$ \\ 
\hline 
500 $k\Omega$ (L4) & 10 $\mu A$ &  &  &  &  \\ 
\hline 
50 $k\Omega$ (L3) & 100 $\mu A$ &  &  &  &  \\ 
\hline 
5 $k\Omega$ (L2) & 1 mA &  &  &  &  \\ 
\hline 
\end{tabular} 
\subsubsection{Hőmérséklet jelleggörbék felvétele}
A mérendő objektum egy 6x6x1cm-es alumínium tömb, melynek
a hátlapjára (6x6cm) egy 47 $\Omega$-os 25 W-os fűtőellenállást
helyeztünk el.
Az érzékelők a tömb felső szélén vannak elhelyezve. Ezzel az
elrendezéssel próbáljuk meg biztosítani, hogy minden érzékelő
közel azonos hőmérsékletet mérjen.
Az alumínium tömb tetején a 6x1 cm-es felületen van 4 féle
érzékelő rögzítve, ezek kivezetéseit találják meg a mérőpanel
előlapján.
A mérőpanel tápfeszültség ellátásának egy része fixen bekötött
(12V és 42V), de az előlapra egy $\pm$15 V-os tápfeszültséget kell
csatlakoztatni.
A hőmérséklet változtatását a beépített kapcsoló változtatásával
tudják elérni.
A maximális hőmérséklet biztonsági okokból kb 50 $^{\circ}C$
A mérést a hőmérséklet beállító kapcsoló minden állásában
végezze el. A hőmérséklet beállítását egy szabályzás végzi. A
kapcsolóval az alapjelet állítják. A stabil hőmérséklet
beállításához kb. 3 percre van szükség. A termikus egyensúly
beállásának elérését a thermisztor ellenállásváltozását figyelve
tudjuk legegyszerűbben nyomon követni. Amikor a thermisztor
ellenállása már nem változik (esetleg csak nagyon lassan)
beálltnak tekinthetjük a termikus egyensúlyt.
A méréshez használjon két darab HM8012-es, egy TR1667/B és
két MX-25201-es típusú digitális multimétert.
Tervezze meg melyik műszerrel érdemes melyik érzékelő kimeneti
jelét mérni! A panel bal oldalán lévő piros banánhüvelyekre csatlakoztassa az
egyik HM8012-es multimétert $^{\circ}C$ mérő üzemmódban, a műszer
által mutatott feszültséget tekintsük a „hiteles” referenica
hőmérsékletnek!
$$$$Mért értékek:$$$$\begin{tabular}{|c|c|c|c|c|}
\hline 
Referencia hőmérséklet [$^{\circ}C$] & Therm [k$\Omega$] & PT100 [$\Omega$] & Hőelem [mV] & IC [mV] \\ 
\hline 
 &  &  &  &  \\ 
\hline 
 &  &  &  &  \\ 
\hline 
 &  &  &  &  \\ 
\hline 
 &  &  &  &  \\ 
\hline 
 &  &  &  &  \\ 
\hline 
 &  &  &  &  \\ 
\hline 
 &  &  &  &  \\ 
\hline 
\end{tabular} $$$$Az ábra az 5. számú mellékletben található meg.
\subsubsection{A szilárd testek felületén mérhető hőmérséklet eloszlásának vizsgálata}
A mérendő objektum felfűtés után a kézi tapintófejes
hőmérsékletmérő műszer segítségével mérje meg az alumínium
tömb közepén és alsó szélén a hőmérsékletét.
Adjon magyarázatot a mért hőmérséklet értékekre.$$$$Mért értékek:$$$$\begin{tabular}{|c|c|}
\hline 
Középen & Alul \\ 
\hline 
 &  \\ 
\hline 
\end{tabular} 
\end{document}