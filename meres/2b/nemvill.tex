\documentclass[10pt,a4paper]{article}
\usepackage[utf8]{inputenc}
\usepackage[magyar]{babel}
\usepackage[T1]{fontenc}
\usepackage{amsmath}
\usepackage{amsfonts}
\usepackage{amssymb}
\usepackage{graphicx}
\begin{document}
\section{16. sz. laboratóriumi mérés}
	\subsection{A mérés célja}
	Egyes nem-villamos fizikai jellemzők (erő, nyomaték, nyomás, mechanikai feszültség)
mérésére alkalmas nyúlásmérő bélyeg fontosabb statikus méréstechnikai jellemzőinek
megállapítása. Egy adott feladatra való alkalmazás megismerése. A mérést “zavaró” jellemzők közül a hőmérsékletváltozás hatásának, mértékének megállapítása, vizsgálata.
	\subsection{Mérési feladatok}
	A mérésben egy befogott rugólap átellenes oldalára ragasztott mérőbélyegekkel állapíthatjuk
meg a lapban keletkező mechanikai feszültség értékét. A rugólap hajlítását csavarorsóval
hozzuk létre.
A hajlítás mértékét - az elmozdulást ($\Delta f$) 1/100 mm-es mérőórával mérjük.
A csavarorsó tengelyében ható F erő L (43 mm) hosszúságú karon végzi a hajlítást, a rugólap
keresztmetszeti méretei: h (1,2 mm) * b (10 mm.)
		\subsubsection{Elmozdulás (hajlítás) - ellenállásváltozás karakterisztika felvétele és kiértékelése}
		A 6. sz. mérőpanelon a felső kivezetések a húzott, az alsók a nyomott bélyeghez csatlakoznak.
Mérje meg mindkét bélyeg ellenállását a rugólap terheletlen és maximális kitérése között 0,5
mm-enként. A mért adatokat foglalja táblázatba és ábrázolja eltolt koordináta rendszerben mm papíron! (Az eltolás mértéke a bélyegek előfeszítés nélküli alap ellenállása.)
Állapítsa meg a húzott és a nyomott bélyegek linearitási és hiszterézis hibáját!$$$$A linearitási hiba megállapítása:$$h_{lin}=\frac{H_{max}}{X_kMT}*100 \%$$Mért adatok húzott bélyeg esetén:$$$$\begin{tabular}{|c|c|c|c|c|c|c|c|}
\hline 
l[mm] & 0 & 0,5 & 1 & 1,5 & 2 & 2,5 & 3 \\ 
\hline 
R[$\Omega$] &  &  &  &  &  &  &  \\ 
\hline 
\end{tabular} $$$$
Mért adatok húzott bélyeg esetén (visszafelé):$$$$\begin{tabular}{|c|c|c|c|c|c|c|c|}
\hline 
l[mm] & 3 & 2,5 & 2 & 1,5 & 1 & 0,5 & 0 \\ 
\hline 
R[$\Omega$] &  &  &  &  &  &  &  \\ 
\hline 
\end{tabular}$$$$
Mért adatok nyomott bélyeg esetén:$$$$\begin{tabular}{|c|c|c|c|c|c|c|c|}
\hline 
l[mm] & 0 & 0,5 & 1 & 1,5 & 2 & 2,5 & 3 \\ 
\hline 
R[$\Omega$] &  &  &  &  &  &  &  \\ 
\hline 
\end{tabular} $$$$ A számolás és az ábra az 1. számú mellékletben található meg.
	\subsubsection{Nyúlásmérő bélyeges negyedhíd vizsgálata!}
	Mérje meg a híd kimeneti feszültségét a rugólap terheletlen és maximális kitérése között 0,5 mm-enként. A hídban történő mérést is mindkét bélyeggel (húzott és nyomott) külön-külön végezze el! (Nekünk csak a húzott bélyeggel kellett.)
Az egyik bélyegnél az elmozdulás csökkentésekor (“visszafelé”) is vegye fel az adatokat!
A hídban lévő R ellenállások értéke: 360 $\pm 1\%$.
Foglalja a mért adatokat táblázatba és rajzolja meg a hídkapcsolás Uki-
(elhajlítás) karakterisztikáját! (Mivel a híd nincsen kinullázva az
ábrázolást eltolt koordináta rendszerben végezze. Az eltolás mértéke a híd
alapállapotban mért kimeneti feszültsége legyen.)
Állapítsa meg a kapcsolás átalakítási tényezőjét!$$$$
Mért adatok húzott bélyeg esetén:$$$$
\begin{tabular}{|c|c|c|c|c|c|c|c|}
\hline 
l[mm] & 0 & 0,5 & 1 & 1,5 & 2 & 2,5 & 3 \\ 
\hline 
$\Delta U$[$mV$] &  & &  &  &  &  &  \\ 
\hline 
\end{tabular} $$$$
Mért adatok húzott bélyeg esetén (visszafelé):$$$$
\begin{tabular}{|c|c|c|c|c|c|c|c|}
\hline 
l[mm] & 3 & 2,5 & 2 & 1,5 & 1 & 0,5 & 0 \\ 
\hline 
$\Delta U$[$mV$] &  &  &  &  &  &  &  \\ 
\hline 
\end{tabular} $$$$
A számolás és az ábra az 2. számú mellékletben található meg.
	\subsubsection{Nyúlásmérő bélyeges félhíd vizsgálata!}
	Mérje meg a híd kimeneti feszültségét a rugólap terheletlen és maximális kitérése között 0,5 mm-enként. Foglalja a mért adatokat táblázatba és rajzolja meg a hídkapcsolás Uki- (elhajlítás) karakterisztikáját! (A karakterisztikát az előző ábrába rajzolja bele!)
Értékelje a kapcsolás érzékenységét a negyedhídhoz képest!
$$$$
Mért adatok húzott bélyeg esetén:$$$$
\begin{tabular}{|c|c|c|c|c|c|c|c|}
\hline 
l[mm] & 0 & 0,5 & 1 & 1,5 & 2 & 2,5 & 3 \\ 
\hline 
$\Delta U$[$mV$] &  &  &  &  &  &  &  \\ 
\hline 
\end{tabular} $$$$ Az ábra az 2. számú mellékletben található meg.
	\subsubsection{A hőmérsékleti hatás vizsgálata!}
	A rugólap terhelését állítsa be a maximális 3 mm-re.
Mérje meg a húzott és a nyomott mérőbélyeg ellenállását.
A mérőpanelba épített hőfokszabályzós fűtőtest segítségével a mérőbélyegek tere (és így a
mérőbélyegek hőmérséklete) kb. 44 $^{\circ}C$-ra beállítható, így a hőmérsékleti hatások
vizsgálhatók.
Kapcsoljon 12 V feszültséget a FŰTÉS feliratú pontra!
Mintegy 15 perc múlva beáll a termikus egyensúly. Ilyenkor a tér hőmérséklete kb. 44$^{\circ}C$.
A hőmérsékletváltozás hatására megváltozik a mérőbélyeg ellenállása.
A tér hőmérsékletének elérését a fűtő áram értékének lecsökkenése jelzi.
A tér felfűtött állapotában ismét mérje meg a húzott és a nyomott mérőbélyeg ellenállását!
Az ellenállás mérést kétféle „polaritással” végezze el! Értékelje a kapott értékeket!
$$$$Húzott bélyeg:$$$$\begin{tabular}{|c|c|c|}
\hline 
 & Fűtés nélkül & Fűtve \\ 
\hline 
Polaritás helyesen &  &  \\ 
\hline 
Ellentétesen &  &  \\ 
\hline 
\end{tabular} 

\end{document}