\documentclass[10pt,a4paper]{article}
\usepackage[utf8]{inputenc}
\usepackage[magyar]{babel}
\usepackage[T1]{fontenc}
\usepackage{amsmath}
\usepackage{amsfonts}
\usepackage{amssymb}
\usepackage{graphicx}
\begin{document}
\section{17. sz. laboratóriumi mérés}
Mérés dátuma: \date{2017.10.09}
	\subsection{A mérés célja}
	A digitális oszcilloszkóp kezelésének többlet funkcióinak
elsajátítása, a kapott mérési eredmények kiértékeléséhez
szükséges szemlélet kialakítása.
	\subsection{Mérési feladatok}
	\subsubsection{Kurzoros mérések}
	Végezzen kurzoros méréseket szinuszos jel frekvenciájának
és amplitúdójának, illetve négyszögjel felfutási idejének
megállapításához, hasonlítsa össze mérési eredményeit az
azonos jelalakon elvégzett automatikus mérések
eredményeivel. (Javasolt jelalakok: 1kHz-es, 1V amplitúdójú
szinuszjel, 1V DC ofszettel, 10kHz-es 1V amplitúdójú
négyszögjel, 100mV DC ofszettel.) Mindenképpen térjen ki a
jelek effektív értékének automatikus mérésére, a kapott
eredményeket hasonlítsa össze a kurzoros mérések alapján
számított értékekkel!$$\begin{tabular}{|c|c|c|}
\hline 
Jelalak & Szinusz & Négyszög \\ 
\hline 
 Amplitúdó (kurzor) &  &  \\ 
\hline 
Amplitúdó (automatikus) &  &  \\ 
\hline 
 Felfutási idő (kurzor) & &  \\ 
\hline 
Felfutási idő (automatikus) &  &  \\ 
\hline 
\end{tabular} $$ 
 \subsubsection{Adatgyűjtési mód}
 	Vizsgálja a függvénygenerátoron a legnagyobb beállítható
frekvenciájú négyszögjel egy periódusát, majd csak a felfutó
élét a három adatgyűjtési üzemmódban! Rajzolja le a látott
jelalakokat!\newpage
	\subsubsection{Bemeneti komparátor működési feltételeinek, működési idejének vizsgálata}
	A LEVEL potencióméterrel a komparálási szintet állítja a SENSE
potencióméterrel pedig a hiszterézis nagyságát.
Adjon az 3. mérőpanelre +5V tápfeszültséget. Állítson be a
függvénygenerátoron 2V csúcsértékű 1kHz frekvenciájú háromszögjelet 0V
egyenfeszültségű összetevővel, a jelet csatlakoztassa a mérőpanel CH1
bemenetére! Kétcsatornás oszcilloszkóp egyik bemenetére a CH1 jelét, másik bemenetére
a KOMP1 komparátor kimeneti jelét csatlakoztatva mérje meg a
következőket: $$$$
	Milyen határok közt tudja állítani a komparálási szintet a LEVEL
potenciométerrel? A SENSE potencióméter a jobb oldali végállásában
legyen! $$$$ $$$$ $$$$
	Mekkora a maximálisan beállítható hiszterézis a SENSE potencióméterrel?
	$$$$
	$$$$
	Az oszcilloszkóp XY üzemmódjának segítségével rajzolja le a
hiszterézises komparátor $U_{ki}-U_{be}$ karakterisztikáját! A SENSE
potencióméter a bal míg a LEVEL potencióméter a jobboldali
végállásban legyen.\newpage
\subsubsection{Komparátor késleltetési idejének vizsgálata}
	Az előbbi kapcsolást felhasználva adjon a CH1 bemenetre 100 kHz-es
négyszögjelet. A négyszög-jel negatív csúcsértéke 0 V, pozitív csúcsértéke 3
V legyen.
Mérje meg a komparátor késleltetési idejét a négyszögjel fel- és lefutó éléhez
képest! 
\end{document}